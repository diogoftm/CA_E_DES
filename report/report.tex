% install texlive-science to enable algorithm package %

\documentclass{article} % Use the report class instead of article
\usepackage{titlesec}
\usepackage{graphicx}
\usepackage{hyperref}
\usepackage{listings}
\usepackage{color}
\usepackage{amsmath} % Add this line to use \text
\usepackage{tabularx}
\usepackage{algorithm}
\usepackage[noend]{algpseudocode}

\definecolor{dkgreen}{rgb}{0,0.6,0}
\definecolor{gray}{rgb}{0.5,0.5,0.5}
\definecolor{mauve}{rgb}{0.58,0,0.82}

\lstset{frame=tb,
  language=Java,
  aboveskip=3mm,
  belowskip=3mm,
  showstringspaces=false,
  columns=flexible,
  basicstyle={\small\ttfamily},
  numbers=none,
  numberstyle=\tiny\color{gray},
  keywordstyle=\color{blue},
  commentstyle=\color{dkgreen},
  stringstyle=\color{mauve},
  breaklines=true,
  breakatwhitespace=true,
  tabsize=3
}

\graphicspath{{./assets/images/}}

\title{%
    \includegraphics[width=0.3\linewidth]{./assets/logo.pdf}\\[20pt]
    \Huge \bfseries Criptografia aplicada \\[10pt]
    \Large EDES
}
\author{Tiago Silvestre, 103554 \\ Diogo Matos, ......}
\date{\today}


\begin{document}



\maketitle

\newpage

\tableofcontents

\clearpage

\section{Introduction}
Introduction ...

\section{Implementation}

\subsection{Encryption}
The encryption using E-DES processes a 64 bit block through a 16 distinct layers of Feistel Networks. 
Each Feistel Network will mix and transform the data using a s-box generated from the key and more
operation in order to mask the determinism of the algorithm.

\subsubsection{Feistel Network}

\begin{algorithm}[H]
  \caption{Feistel Network}
  \begin{algorithmic}[1]

  \State $l[4] \gets \{in[0], in[1], in[2], in[3]\}$
  \State $r[4] \gets \{in[4], in[5], in[6], in[7]\}$
  \State $rf \gets$ \Call{f}{$r, sbox$}
  \State $result[8] \gets$ \textbf{new} $uint8\_t[8]$
  \For{$i \gets 0$ to $3$}
    \State $result[i + 4] \gets l[i] \oplus rf[i]$
    \State $result[i] \gets r[i]$
  \EndFor
  \State \Return $result$
  \end{algorithmic}
\end{algorithm}

It starts by spliting the 64 bit block into two 32 bit blocks.
The less significant one is processed by a \textit{f} function that uses a S-Box (see \autoref{sboxgen}) 
generated from the key to transform deterministically each 2 bytes of the input into other 2 bytes.
The output of the network is the less significant 32 bit block appended to the result of a xor operation between 
the most significant block and the output of the \textit{f} function.

\subsubsection{\textit{f} function}

\begin{algorithm}[H]
  \caption{\textit{f} function}
\begin{algorithmic}[1]
  \State $index \gets in[3]$
  \State $out \gets$ \textbf{new} $uint8\_t[4]$
  \State $out[0] \gets sbox[index]$
  \State $index \gets (index + in[2]) \mod 256$
  \State $out[1] \gets sbox[index]$
  \State $index \gets (index + in[1]) \mod 256$
  \State $out[2] \gets sbox[index]$
  \State $index \gets (index + in[0]) \mod 256$
  \State $out[3] \gets sbox[index]$
  \State \textbf{return} $out$
\end{algorithmic}
\end{algorithm}

A \textit{f} function is used by a Feistel Network in order to create confusion on the input block.
This function recieves a block of 32 bits and the S-Box to used. The most significant byte of the output is 
the result of the S-Box for the less significant byte. For the next byte (2nd most significant) in the 
output is the result of the S-Box for an input with the sum modulo 256 of the two input bytes in the higher memory addresses.
This process goes on for the 4 bytes. See the algorithm above for a more implementation-like algorithm representation.

\subsubsection{Implmentation notes}
The libary developed in C++ to the user (developer) is not that different from the open-ssl inplementation.
First, create an instance of the class, then set the key and, at last, encrypt. But it has one major difference,
the encrypt function is able to handle any data multiple of the block size (8 bytes). 
The programs that use EDES library does't need to iterate over the blocks and encrypt one by one.

\subsection{Decryption}

\begin{algorithm}[H]
  \caption{Decryption Feistel Network}
  \begin{algorithmic}[1]
      \State $l[4] \gets \{in[0], in[1], in[2], in[3]\}$
      \State $r[4] \gets \{in[4], in[5], in[6], in[7]\}$
      \State $rf \gets$ \Call{f}{$r, sbox$}
      \State $result \gets$ \textbf{new} $uint8\_t[8]$
      \For{$i \gets 0$ to $3$}
          \State $result[i + 4] \gets l[i] \oplus rf[i]$
          \State $result[i] \gets r[i]$
      \EndFor
      \State \textbf{return} $result$
  \end{algorithmic}
  \end{algorithm}

The decryption is very simular to the encryption. Ther are only two differences. The S-Boxs are used in reverse order. In the Feistel Networks that does
its operations in reverse. So, It starts by spliting the 64 bit block of the ciphertext into two 32 bit blocks.
The most significant one is processed by the \textit{f} function.
The output of the network is the most significant 32 bit block appended to the result of a xor operation between 
the less significant block and the output of the \textit{f} function.

\subsection{S-Boxes Generation}
\label{sboxgen}
For the purpose of generating S-Boxes, the process is divided in three phases:
\begin{enumerate}
  \item Generate a derived array of 8192 bytes from the a 32 byte key
  \item Mix a sorted array based on the derived array generated
  \item Fill S-Boxes bytes with uniformely distributed values ranging from 0 up to 255 based on the shuffled array generated in step 2. 
\end{enumerate}

\subsubsection{Generate a derived array from the key}
This step has an input of a 32 byte key ($K$) and it should return a 8192 byte array ($D$).
In order to calculate derived array values the following algorithm should be used:

\begin{equation}
  D_0 = SHA256(K)
\end{equation}

\begin{equation}
  D_i = \text{SHA256}(K, D_{i-1}), \text{ for } i \in \{1, \ldots, 255\}
\end{equation}

\subsubsection{Mix sorted array}
In this phase a sorted array ($A$) is shuffled based on the derived array.
The sorted array should be sorted in the following way:

\begin{equation}
  A[i] = i \text{ for } i \in \{0, \ldots, 4095\}
\end{equation}

Then the array should be iterated from $0$ up to $4094$ and for each iteration pairs of two bytes (mod $4096$) from $D$ should be picked and swapped with current iteration position.

\begin{algorithm}[H]
  \caption{Array Initialization and Shuffling}
  \begin{algorithmic}[1]
  \State Initialize array $A$ with $4096$ elements
  \For{$i \gets 0$ \textbf{to} $4095$}
      \State $A[i] \gets i$
  \EndFor
  
  \For{$i \gets 0$ \textbf{to} $4094$}
      \State $pos \gets ((D[2 \cdot (i+1) - 1] \ll 8) + D[2 \cdot i]) \mod 4096$
      \State Swap($A$, $i$, $pos$)
  \EndFor
  \end{algorithmic}
\end{algorithm}


\subsubsection{Fill S-boxes bytes}

\begin{algorithm}[H]
  \caption{Fill S-boxes}
  \begin{algorithmic}[1]

    \State $currentValue \gets 0$
    \State $currentIteration \gets 0$

    \For{$i \gets 0$ \textbf{to} $4095$}
      \State $position \gets Shuffled[i]$
      \State $box\_number \gets \lfloor position / 256 \rfloor$
      \State $box\_position \gets position \mod 256$
      \State $Box[box\_number][box\_position] \gets currentValue$
      \State $currentIteration \gets currentIteration + 1$
      \If{$currentIteration \geq 16$}
          \State $currentIteration \gets 0$
          \State $currentValue \gets currentValue + 1$
      \EndIf
    \EndFor

  \end{algorithmic}
\end{algorithm} 

\subsubsection{Speed improvement methodologies used}

After implementing the E-DES libraries (C++ and Python), we made some test using the Speed program (\href{speed}) in order to compare the performance of our implementation
of E-DES compared to the implementation of DES by famous libraries. E-DES should be a faster algorithm because of its architecture. But, as is well know, the code metters
when it comes to time performance. So, based on the first implementation, we deployed some thecniques to refactor the code to be faster.

For the C++ implementation, the first test have shown that our E-DES was ten times slower tha open ssl's DES. The Python implementation was...

Apart from basic good pratices, we used the following thecniques:
\begin{enumerate}
  \item Code profiling - the usage of a profiler, we used gprof for c++ and..., can pinpoint the critical places of the code. By giving metrics such as the number of 
  times a method was called and the time it spend executing it, it is possible to know where to optimize.
  \item Loop unrolling - by removing loops or by reducing the number of iterations can lead to performance improvements. In some cases it was possible to remove the 'for'
  statements because it the number of iteration was hardcoded, by doing that at least the program doesn't need to perform jump instructions to perform the cycle.
  \item Code batching - method calls are expensive, batching the code into less methods it can lead to performance improvements. This thecniques usually causes the code 
  to be less readable. Batching together the f function and the Feistel Network into the block processing method had a great positive impact.
  \item Hoisting - by removing some part of the code from inside of a loop to outside can lead to performance improvements, particularly when there's variables being
  declared inside the loop
\end{enumerate}

\section {Discussion}

\subsection{Encryption}
Explain why encryption works

\subsection{Decryption}
Explain why decryption works

\subsection{S-Boxes generation}

Considering that S-boxes occupy 4096 bytes, it's essential to have a sufficiently long byte vector with data that is uniformly distributed to effectively populate each S-box. To generate this vector (array), the SHA-256 cryptographic hash function was employed. This approach ensures a uniform distribution of data, and because SHA-256 is a one-way function, it is practically impossible to reverse-engineer the key from these values.

The reason for hashing all the values with the key is to prevent the recovery of the right side of the vector if a single hash value is discovered. This additional layer of security ensures that, without the key, generating the remaining bytes becomes infeasible.

In the second step, a sorted array containing values from 0 to 4096 is subjected to a shuffling process. There are two primary reasons why the array begins in a sorted state with unique values:

\begin{enumerate}
\item To ensure determinism in the algorithm, it's crucial for the array to start in a unique configuration, guaranteeing consistent output for the same input.
\item The final array should encompass values from 0 to 4095 without any repetitions. This initial state maintains this property, and subsequent swaps do not compromise this characteristic.
\end{enumerate}

During the shuffling process, two bytes are combined, and modulo 4096 is applied. This operation is performed because the array contains only 4096 positions. It's worth noting that this calculation doesn't impact the distribution of data due to the compatibility between the values. Specifically, we have chosen pairs of 2 bytes from the array ($2^{16}$ possibilities), which is a multiple of $2^{10}$. A similar rationale applies when selecting 2 bytes from the 32 bytes generated by SHA-256. For further insights into this concept, you can refer to this \href{https://crypto.stackexchange.com/a/21010}{source}.

The final shuffled array can be considered unbiased in accordance with the \href{https://en.wikipedia.org/wiki/Fisher%E2%80%93Yates_shuffle}{Fisher-Yates shuffle} which guarantees in this case that 4095 pseudorandom swaps produce an unbiased permutation. 

In the final phase of the algorithm, S-boxes are filled according to a shuffled array. This guarantees that every value ranging from 0 to 255 is uniformly distributed throughout the S-boxes. It iterates 4096 times, and every 16 iterations, the value is incremented, meaning that there are 256 different values occurring 16 times each.
\section{Statistics}
Write about algorithm performance ... (compare with DES)
Show some graphs about avg time, fastest time etc ...

\section {Aplication}
\subsection{Encryption}
This applicattion performs the following steps:
\begin{enumerate}
  \item Read the plaintext to encrypted from the stdin and password given by the user as an argument;
  \item Derive the 256 bit key from the password to be used for E-DES or DES:
  \item Pad the plaintext using PKCS\#7;
  \item Encrypt the data with padding. It uses E-DES or DES based on the usage of the '-s' flag by the user;
  \item Encode the ciphertext to base64:
  \item Print result in the stdout.
\end{enumerate}

\subsection{Decryption}
This applicattion performs the following steps:
\begin{enumerate}
  \item Read the ciphertext to encrypted from the stdin and password given by the user as an argument;
  \item Derive the 256 bit key from the password to be used for E-DES or DES:
  \item Decrypt the data. It uses E-DES or DES based on the usage of the '-s' flag by the user;
  \item Unpad the plaintext (PKCS\#7);
  \item Decode the ciphertext from base64:
  \item Print result in the stdout.
\end{enumerate}

\subsection{Speed}
\label{speed}

\end{document}