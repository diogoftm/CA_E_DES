% install texlive-science to enable algorithm package %

\documentclass{article} % Use the report class instead of article
\usepackage{titlesec}
\usepackage{graphicx}
\usepackage{hyperref}
\usepackage{listings}
\usepackage{color}
\usepackage{amsmath} % Add this line to use \text
\usepackage{tabularx}
\usepackage{algorithm}
\usepackage{tcolorbox}
\usepackage[noend]{algpseudocode}

\definecolor{dkgreen}{rgb}{0,0.6,0}
\definecolor{gray}{rgb}{0.5,0.5,0.5}
\definecolor{mauve}{rgb}{0.58,0,0.82}

\lstset{frame=tb,
  language=Java,
  aboveskip=3mm,
  belowskip=3mm,
  showstringspaces=false,
  columns=flexible,
  basicstyle={\small\ttfamily},
  numbers=none,
  numberstyle=\tiny\color{gray},
  keywordstyle=\color{blue},
  commentstyle=\color{dkgreen},
  stringstyle=\color{mauve},
  breaklines=true,
  breakatwhitespace=true,
  tabsize=3
}

\graphicspath{{./assets/images/}}

\title{%
    \includegraphics[width=0.3\linewidth]{./assets/logo.pdf}\\[20pt]
    \Huge \bfseries Criptografia aplicada \\[10pt]
    \Large EDES
}
\author{Tiago Silvestre, 103554 \\ Diogo Matos, 102848}
\date{\today}


\begin{document}



\maketitle

\newpage

\tableofcontents

\clearpage

\section{Introduction}
Data Encryption Standard (DES) was developed in the early 70's and was considered secure and became quite popular. It uses a 56 bit key and block sizes of 64 bits performing 
16 round of Feistel Networks and permutations in the beginning and at the end. The secrecy behind the values used in used in the S-Boxes led to concerns of possible
hiden trapdoors and, most importantly, the key size has been too small for more than two decades. Its lack of security can be solved by the means of cipher modes. Another
aspect that is not desirable is how slow the DES is compared to its competition.

In this project we introduce a new cipher called Enhanced Data Encryption Standard (E-DES) that mainly aims to twist DES in order to be more secure and faster. 
E-DES uses a 256 bit key, maintains the block size, removes both permutation boxes, it does not use sub-keys either, and the creation of the S-Boxes depends only on the key. In this report we go into 
further detail about the main aspects of E-DES, we give a brief overview of our implementation and discuss the results.

\section{Implementation}
\begin{tcolorbox}
  [width=\linewidth, colback=white!95!black, boxrule=0pt]
  C++    - /cpp/EDES.cpp

  Python - /python/EDES.py
\end{tcolorbox}


\subsection{Encryption}
The encryption using E-DES processes a 64 bit block through 16 distinct layers of Feistel Networks. 
Each Feistel Network will mix and transform the data using a S-Box generated from the key and more
operation in order to mask the determinism of the algorithm. 

The main operations are done in a Feistel Network and \textit{f} function.

\subsubsection{Feistel Network}

\begin{algorithm}[H]
  \caption{Feistel Network}
  \begin{algorithmic}[1]

  \State $l[4] \gets \{in[0], in[1], in[2], in[3]\}$
  \State $r[4] \gets \{in[4], in[5], in[6], in[7]\}$
  \State $rf \gets$ \Call{f}{$r, sbox$}
  \State $result[8] \gets$ \textbf{new} $uint8\_t[8]$
  \For{$i \gets 0$ to $3$}
    \State $result[i + 4] \gets l[i] \oplus rf[i]$
    \State $result[i] \gets r[i]$
  \EndFor
  \State \Return $result$
  \end{algorithmic}
\end{algorithm}

It starts by spliting the 64 bit block into two 32 bit blocks.
The less significant one is processed by a \textit{f} function that uses a S-Box (see \autoref{sboxgen}) 
generated from the key to transform deterministically each 2 bytes of the input into other 2 bytes.
The output of the network is the less significant 32 bit block appended to the result of a xor operation between 
the most significant block and the output of the \textit{f} function.

\subsubsection{\textit{f} function}

\begin{algorithm}[H]
  \caption{\textit{f} function}
\begin{algorithmic}[1]
  \State Initialize array $in$ with 4 bytes (half block)
  \State $index \gets in[3]$
  \State $out \gets$ \textbf{new} $uint8\_t[4]$
  \State $out[0] \gets sbox[index]$
  \State $index \gets (index + in[2]) \mod 256$
  \State $out[1] \gets sbox[index]$
  \State $index \gets (index + in[1]) \mod 256$
  \State $out[2] \gets sbox[index]$
  \State $index \gets (index + in[0]) \mod 256$
  \State $out[3] \gets sbox[index]$
  \State \textbf{return} $out$
\end{algorithmic}
\end{algorithm}

A \textit{f} function is used by a Feistel Network in order to create confusion on the input block.
This function recieves a block of 32 bits and the S-Box to used. The most significant byte of the output is 
the result of the S-Box for the less significant byte. For the next byte (2nd most significant) in the 
output is the result of the S-Box for an input with the sum modulo 256 of the two input bytes in the higher memory addresses.
This process goes on for the 4 bytes. See the algorithm above for a more implementation-like algorithm representation.

\subsection{Decryption}

\begin{algorithm}[H]
  \caption{Decryption Feistel Network}
  \begin{algorithmic}[1]
      \State Initialize array $in$ with 8 bytes (a block)
      \State $l[4] \gets \{in[0], in[1], in[2], in[3]\}$
      \State $r[4] \gets \{in[4], in[5], in[6], in[7]\}$
      \State $rf \gets$ \Call{f}{$r, sbox$}
      \State $result \gets$ \textbf{new} $uint8\_t[8]$
      \For{$i \gets 0$ to $3$}
          \State $result[i + 4] \gets l[i] \oplus rf[i]$
          \State $result[i] \gets r[i]$
      \EndFor
      \State \textbf{return} $result$
  \end{algorithmic}
  \end{algorithm}

The decryption is very similar to the encryption. There are only two differences. The S-Boxes are used in reverse order and the Feistel Networks does
its operations in reverse. So, the network starts by spliting the 64 bit block of the ciphertext into two 32 bit blocks,
the most significant one is processed by the \textit{f} function and the output of the network is the most significant 32 bit block appended to the result of a xor operation between 
the less significant block and the output of the \textit{f} function.

\subsection{S-Boxes Generation}
\label{sboxgen}
For the purpose of generating S-Boxes, the process is divided in three phases:
\begin{enumerate}
  \item Generate a derived array of 8192 bytes from the a 32 byte key
  \item Mix a sorted array based on the derived array generated
  \item Fill S-Boxes bytes with uniformely distributed values ranging from 0 up to 255 based on the shuffled array generated in step 2. 
\end{enumerate}

\subsubsection{Generate a derived array from the key}
This step has an input of a 32 byte key ($K$) and it should return a 8192 byte array ($D$).
In order to calculate derived array values the following algorithm should be used:

\begin{equation}
  D_0 = SHA256(K)
\end{equation}

\begin{equation}
  D_i = \text{SHA256}(K, D_{i-1}), \text{ for } i \in \{1, \ldots, 255\}
\end{equation}

\subsubsection{Mix sorted array}
In this phase a sorted array ($A$) is shuffled based on the derived array.
The sorted array should be sorted in the following way:

\begin{equation}
  A[i] = i \text{ for } i \in \{0, \ldots, 4095\}
\end{equation}

Then the array should be iterated from $0$ up to $4094$ and for each iteration pairs of two bytes (mod $4096$) from $D$ should be picked and swapped with current iteration position.

\begin{algorithm}[H]
  \caption{Array Initialization and Shuffling}
  \begin{algorithmic}[1]
  \State Initialize array $A$ with $4096$ elements
  \For{$i \gets 0$ \textbf{to} $4095$}
      \State $A[i] \gets i$
  \EndFor
  
  \For{$i \gets 0$ \textbf{to} $4094$}
      \State $pos \gets ((D[2 \cdot (i+1) - 1] \ll 8) + D[2 \cdot i]) \mod 4096$
      \State Swap($A$, $i$, $pos$)
  \EndFor
  \end{algorithmic}
\end{algorithm}


\subsubsection{Fill S-Boxes bytes}

\begin{algorithm}[H]
  \caption{Fill S-boxes}
  \begin{algorithmic}[1]

    \State $currentValue \gets 0$
    \State $currentIteration \gets 0$

    \For{$i \gets 0$ \textbf{to} $4095$}
      \State $position \gets Shuffled[i]$
      \State $box\_number \gets \lfloor position / 256 \rfloor$
      \State $box\_position \gets position \mod 256$
      \State $Box[box\_number][box\_position] \gets currentValue$
      \State $currentIteration \gets currentIteration + 1$
      \If{$currentIteration \geq 16$}
          \State $currentIteration \gets 0$
          \State $currentValue \gets currentValue + 1$
      \EndIf
    \EndFor

  \end{algorithmic}
\end{algorithm} 

\section {Discussion}

\subsection{S-Boxes generation}
\label{sboxgen}
Considering that S-boxes occupy 4096 bytes, it's essential to have a sufficiently long byte vector with data that is uniformly distributed to effectively populate each S-box. To generate this vector (array), the SHA-256 cryptographic hash function was employed. This approach ensures a uniform distribution of data, and because SHA-256 is a one-way function, it is practically impossible to reverse-engineer the key from these values.

The reason for hashing all the values with the key is to prevent the recovery of the right side of the vector if a single hash value is discovered. This additional layer of security ensures that, without the key, generating the remaining bytes becomes infeasible.

In the second step, a sorted array containing values from 0 to 4096 is subjected to a shuffling process. There are two primary reasons why the array begins in a sorted state with unique values:

\begin{enumerate}
\item To ensure determinism in the algorithm, it's crucial for the array to start in a unique configuration, guaranteeing consistent output for the same input.
\item The final array should encompass values from 0 to 4095 without any repetitions. This initial state maintains this property, and subsequent swaps do not compromise this characteristic.
\end{enumerate}

During the shuffling process, two bytes are combined, and modulo 4096 is applied. This operation is performed because the array contains only 4096 positions. It's worth noting that this calculation doesn't impact the distribution of data due to the compatibility between the values. Specifically, we have chosen pairs of 2 bytes from the array ($2^{16}$ possibilities), which is a multiple of $2^{10}$. A similar rationale applies when selecting 2 bytes from the 32 bytes generated by SHA-256. For further insights into this concept, you can refer to this \href{https://crypto.stackexchange.com/a/21010}{source}.

The final shuffled array can be considered unbiased in accordance with the \href{https://en.wikipedia.org/wiki/Fisher%E2%80%93Yates_shuffle}{Fisher-Yates shuffle} which guarantees in this case that 4095 pseudorandom swaps produce an unbiased permutation. 

In the final phase of the algorithm, S-boxes are filled according to a shuffled array. This guarantees that every value ranging from 0 to 255 is uniformly distributed throughout the S-boxes. It iterates 4096 times, and every 16 iterations, the value is incremented, meaning that there are 256 different values occurring 16 times each.


\subsection{Security}
There is no straightforward method to evaluate a cryptographic algoritm such as a cipher. But there are some metrics that are usually indicative of its quality, 
we explore some of them here.

\paragraph{Bits of security}
The number of n bits of security is good interesting measure since an attacker would need to perform $2^{n}$ operations to crack the cipher. In the case in hand, an operation
is a decryption using an arbitrary key. This measure shows how good is the cipher agains a brute force attack.
Nowadays any value from 128 forward is considered secure. E-DES has 256-bits of security, DES has only 56.

\paragraph{Number of confusion bits}
Confusion is one of the two most important characteristics of a cipher. It defines the relation between the key and the produced ciphertext, the outout bits should be 
highly dependent on the key. Confusion is inducted by the S-Boxes, the discussion at \autoref{sboxgen} shows good signs for the properties of the S-Boxes. 
In order to measure confusion, despite it doesn't prove cipher security alone, is to calculate, on average, how many bits change in the 
ciphertext when changing just one bit of the key. We call this the number of confusion bits.

Idealy the number of confusion bits should be block\_size/2. For a 64 bit block cipher (such as E-DES), 32 is the number to ideally expect. A script was developed in order
to test this that can be found in /cpp/test.cpp. The results show that the number of confusion bits for E-DES is 32.

\paragraph{Number of diffusion bits}
Difusion is a desirable property in a cipher because one small change in the plaintext will lead to a big change in the ciphertext. 
The way to measure diffusion is done changing just one bit in the plaintext and compare to the ciphertext generated for the original plaintext. By counting the average number
of changed bits, we're able to have a metric for diffusion. 

Idealy the number of diffusion bits should be block\_size/2. For a 64 bit block cipher (such as E-DES), 32 is the number to ideally expect. A script was developed in order
to test this that can be found in /cpp/test.cpp. The results show that the number of diffusion bits for E-DES is 32.


\subsubsection{Speed}
\label{speedinp}
\paragraph{Speed improvement methodologies used}

Apart from basic good pratices, we used the following thecniques:
\begin{enumerate}
  \item Code profiling - the usage of a profiler, we used gprof for c++, can pinpoint the critical places of the code. By giving metrics such as the number of 
  times a method was called and the time it spend executing it, it is possible to know where to optimize;
  \item Loop unrolling - by removing loops or by reducing the number of iterations can lead to performance improvements. In some cases it was possible to remove the 'for'
  statements because it the number of iteration was hardcoded, by doing that at least the program doesn't need to perform jump instructions to perform the cycle;
  \item Code batching - method calls are expensive, batching the code into less methods it can lead to performance improvements. This thecniques usually causes the code 
  to be less readable. Batching together the f function and the Feistel Network into a block processing method had a great positive impact;
  \item Hoisting - by removing some part of the code from inside of a loop to outside can lead to performance improvements, particularly when there's variables being
  declared inside the loop.
\end{enumerate}

\paragraph{Speed comparison}

After implementing the E-DES libraries (C++ and Python), we made some test using the speed program in order to compare the performance of our implementation
of E-DES compared to the implementation of DES by famous libraries. E-DES should be a faster algorithm because of its architecture. But, as is well know, the code metters
when it comes to time performance. So, based on the first implementation, we deployed some thecniques to refactor the code to be faster.

For the C++ implementation we applied the thecniques enumerated before in order to improve the speed of encryption and decryption. The Python implementation was developed only to prove that
the algorithm can be implemented in any language. In a production scenario, the Python's E-DES library would only be a interface to the C++ implementation and that's  why the comparison
between our implementation of E-DES and DES from libraries commonly used it's not appropriate.

\begin{figure}[h]
  \center
  \includegraphics[scale=0.2]{assets/speed.png}
  \caption{Speed.cpp output.}
\end{figure}

Our C++ E-DES is around 10\% slower than Open SSL's DES, when tested in a laptop witha a AMD Ryzen 7 7735HS and 16Gb of RAM. E-DES encryption values almost don't fluctuate but
DES times go up and down even sometimes surpassing E-DES's on average. This results are encouraging because we're comparing our proof-of-concept implementation of E-DES with Open 
SSL's very optimized production DES library.

\section{Conclusions}

At the architectural level, E-DES evolves DES by increasing the key length, by simplifying the encryption/decryption process and by generating S-Boxes that depend only
on the key. We proved its feasibility by developing two implementation of the algorithm and by performing various tests. E-DES have shown to be 
a viable cipher.

\end{document}
